%%%%%%%%%%%%%%%%%%%%%%%%%%%%%%%%%%%%%%%%%%%%%%%%%%%%%%%%%%%%%%%%%%%%%%%%%%%%%%%%
%2345678901234567890123456789012345678901234567890123456789012345678901234567890
%        1         2         3         4         5         6         7         8
% THESIS CONCLUSIONS
\def\baselinestretch{1}
\chapter{Conclusions}
\label{chap:conclusions}
\ifpdf
    \graphicspath{{Conclusions/Figures/PNG/}{Conclusions/Figures/PDF/}{Conclusions/Figures/}}
\else
    \graphicspath{{Conclusions/Figures/EPS/}{Conclusions/Figures/}}
\fi
\def\baselinestretch{1.66}

This project resulted in a complete integration between different heterogeneous modules. The goal is to stabilize and perform a number of tasks with an IRIS quadcopter with the help of position feedback given by a motion capture system. The course of this project is identified by important steps, or milestones at this point, which reflect the followed plan and highlight important informations that the reader should understand. 

A crucial phase is the integration of different systems which is the necessary condition needed to perform any kind of research. This aspect is treated in this thesis since it is the fist time that an experimental setup, intended for the investigation of aerial vehicle technology, was used in the University of Genoa. In particular