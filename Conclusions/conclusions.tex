%%%%%%%%%%%%%%%%%%%%%%%%%%%%%%%%%%%%%%%%%%%%%%%%%%%%%%%%%%%%%%%%%%%%%%%%%%%%%%%%
%2345678901234567890123456789012345678901234567890123456789012345678901234567890
%        1         2         3         4         5         6         7         8
% THESIS CONCLUSIONS
\def\baselinestretch{1}
\chapter{Conclusions}
\label{chap:conclusions}
\ifpdf
    \graphicspath{{Conclusions/Figures/PNG/}{Conclusions/Figures/PDF/}{Conclusions/Figures/}}
\else
    \graphicspath{{Conclusions/Figures/EPS/}{Conclusions/Figures/}}
\fi
\def\baselinestretch{1.66}

This project resulted in a complete integration between different heterogeneous modules. The goal is to stabilize and perform a number of tasks with an IRIS quadcopter with the help of position feedback given by a motion capture system. The course of this project is identified by important steps, or milestones at this point, which reflect the followed plan and highlight important informations that the reader should understand. 

A crucial phase is the integration of different systems which is the necessary condition needed to perform any kind of research. This aspect is treated in this thesis since it is the fist time that an experimental setup, intended for the investigation of aerial vehicle technology, was used in the University of Genoa. In particular, by tweaking the reference frames, the system is independent from geographical landmarks like the North-South line and moreover is able to localize himself in a controlled environment. Key point of this phase is the adaptation of the two coordinate system: mocap and earth frame. I would like to stress that the integration phase lead to the discovery of a number of bugs which, after being corrected, were pushed to the original PX4 repository and accepted by the administrators. This made me a contributor of the project PX4/PixHawk and permitted two PX4 users to continue their work. The integration of the on board estimator with a position feedback signal is the final step of this first phase. With the trick of faking the earth magnetic field, the system is free from any external source other than the mocap and we can place earth frame wherever is convenient.

I was involved in the relocation of the equipment in the new laboratory, thus a physical integration was also performed solving a number of logistic issues . Thanks to the setup of this new environment, a new study platform is available from now on giving the possibility to expand the research field.

A second milestone is the stabilization of the quadrotor on a position set point. This phase was possible only after modifying the structure of the on board position controller, since it was designed only to fly with GPS, and after tuning controller parameters specifically for IRIS. 

The most important milestone is the designing, development and testing of the software architecture. A simply behavioral system takes place of the so called ground station. This software is intended for indoor flight and specifically designed for this application. Nevertheless it can potentialy drive any setup with an OptiTrack based motion capture and a MAVLink based quadrotor. 