%%%%%%%%%%%%%%%%%%%%%%%%%%%%%%%%%%%%%%%%%%%%%%%%%%%%%%%%%%%%%%%%%%%%%%%%%%%%%%%%
%2345678901234567890123456789012345678901234567890123456789012345678901234567890
%        1         2         3         4         5         6         7         8
% THESIS ABSTRACT

% Use the following style if the abstract is long:
%\begin{abstractslong}
%\end{abstractslong}

\begin{abstracts}

 A quadcopter, also known as quadrotor, is a helicopter with four rotors. The rotors are directed upwards and they are placed in a symmetric formation with equal distance from the center of mass. The quadcopter is controlled by adjusting the angular velocities of the rotors which are driven by electric motors. An on board autopilot, namely PiXHawk with PX4 software, packs all the interfaces for sensors, motor controllers and radio antennas. It also manages the control, state estimation and signals processing. The goal of this master thesis is to develop algorithms for UAV trajectory planning and execution (as well as the related SW components), using the motion capture system as the source of the UAV position feedback. A 3D Robotics IRIS quadrotor is used. After an introduction to the components, the integration between them both at the hardware and the software levels is presented. The model of the IRIS is briefly derived and explained as well as the control techniques involved in the open source autopilot. Moreover a software architecture is presented where a task-based program is able to read a set of tasks from a list and let the robot execute them sending position set points through a radio link. At the end an algorithm for tracking and landing on a mobile platform is explained and the overall results are presented.

\end{abstracts}